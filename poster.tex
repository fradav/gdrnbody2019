% Gemini theme
% https://github.com/anishathalye/gemini

\documentclass[final]{beamer}

% ====================
% Packages
% ====================


\usepackage[T1]{fontenc}
\usepackage{lmodern}
\usepackage[orientation=portrait,size=a0,scale=1.0]{beamerposter}
\usetheme{gemini}
\usecolortheme{gemini}
\usepackage{amsmath}
\usepackage{wrapfig}
\usepackage{graphicx}
\usepackage{booktabs}
\usepackage{multicol}
\usepackage{natbib}
\usepackage{hyperref}
\usepackage{neuralnetwork}

\renewcommand{\bibpreamble}{\setlength{\columnsep}{30pt}\begin{multicols}{3}}
\renewcommand{\bibpostamble}{\end{multicols}}

\newcommand{\x}[2]{$x_{2}$}
\newcommand{\y}[2]{$\hat{y}_{2}$}
\newcommand{\hlayer}[2]{\small $h^{(#1)}_#2$}

% ====================
% Lengths
% ====================

% If you have N columns, choose \sepwidth and \colwidth such that
% (N+1)*\sepwidth + N*\colwidth = \paperwidth
\newlength{\sepwidth}
\newlength{\colwidth}
\setlength{\sepwidth}{0.0333\paperwidth}
\setlength{\colwidth}{0.4\paperwidth}

\newcommand{\separatorcolumn}{\begin{column}{\sepwidth}\end{column}}

% ====================
% Title
% ====================

\title{
  \noindent
\makebox[0pt][l]{}%
\makebox[\textwidth][c]{Deep Learning and quantum many-body problem}\\%
% \makebox[0pt][r]{\raisebox{1.5ex}{\footnotesize{Code and binaries available at \url{http://github.com/diyabc/abcranger}\hspace{2cm}}}}\\
\LARGE{Further opening the black box of Deep Learning via quantum many-body problem, survey and perspectives}
}
% \title{\RaggedLeft \footnotesize{Code and binaries available at \url{http://github.com/diyabc/abcranger}}\\
% \Centering \Huge{AbcRanger}\\
%  \LARGE{A fast and scalable random forest library for ABC model choice and parameter estimation}}

\author{F.-D. Collin \inst{1}}

\institute[shortinst]{\inst{1} Université de Montpellier, CNRS, IMAG UMR 5149}

% \author{Alyssa P. Hacker \inst{1} \and Ben Bitdiddle \inst{2} \and Lem E. Tweakit \inst{2}}

% \institute[shortinst]{\inst{1} Some Institute \samelineand \inst{2} Another Institute}

% ====================
% Body
% ====================

\begin{document}

\begin{frame}[t]


  \begin{columns}[t]
    \separatorcolumn

    \begin{column}{\colwidth}

      \begin{block}{Generalities on Deep Learning}
        \begin{figure}[ht]
          \resizebox{\columnwidth}{!}{%
%          \begin{neuralnetwork}[height=3pt,layertitleheight=60pt,layerspacing=200pt,nodespacing=50pt,nodesize=30pt]
            \begin{neuralnetwork}[height=4,layertitleheight=60pt,layerspacing=200pt,nodespacing=50pt,nodesize=40pt]
            \inputlayer[count=3, bias=true, title=Input layer, text=\x]
            \hiddenlayer[count=3, bias=false, title=Hidden layer 1, text=\hlayer] \linklayers
            \hiddenlayer[count=4, bias=false, title=Hidden layer 2, text=\hlayer] \linklayers
            \hiddenlayer[count=3, bias=false, title=Hidden layer 3, text=\hlayer] \linklayers
            \outputlayer[count=2, title=Output layer, text=\y] \linklayers
        \end{neuralnetwork}
          }
          \caption{Example of a \emph{feedforward} neural network. Each layer is a tensor operation taking $f_i$ the previous layer output tensor as input. The final function $f(\boldsymbol{x})=f^{(3)}\left(f^{(2)}\left(f^{(1)}(\boldsymbol{x})\right)\right)$ is the composition of the tensor operations.}
        \end{figure}

        \cite{goodfellow2016deep} illustrates this.
      \end{block}


      \begin{block}{Tensor Networks on nbody quantum problem}


      \end{block}

      \begin{alertblock}{DL/nbody coupling}
      \end{alertblock}

    \end{column}

    \separatorcolumn

    \begin{column}{\colwidth}

      \begin{block}{Fractality}

      \end{block}

      \begin{block}{Hamiltonian}



      \end{block}

      \begin{alertblock}{Perspectives}

      \end{alertblock}



    \end{column}

    \separatorcolumn
  \end{columns}

  \begin{block}{\justifying References}
    \nocite{*}
    \footnotesize{\bibliographystyle{unsrt}\bibliography{poster}}

  \end{block}

\end{frame}

\end{document}
